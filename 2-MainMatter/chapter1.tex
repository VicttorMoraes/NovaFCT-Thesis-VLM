%!TEX root = ../template.tex
%%%%%%%%%%%%%%%%%%%%%%%%%%%%%%%%%%%%%%%%%%%%%%%%%%%%%%%%%%%%%%%%%%%
%% chapter1.tex
%% NOVA thesis document file
%%
%% Chapter with introduction
%%%%%%%%%%%%%%%%%%%%%%%%%%%%%%%%%%%%%%%%%%%%%%%%%%%%%%%%%%%%%%%%%%%

\typeout{NT FILE chapter1.tex}%

\chapter{Introduction}
\label{cha:introduction}

\prependtographicspath{{\DIRFIGURES/Covers/}}

% epigraph configuration
\epigraphfontsize{\small\itshape}
\setlength\epigraphwidth{12.5cm}
\setlength\epigraphrule{0pt}


\section{Context and motivation}
\label{sec:a_bit_of_history}

\ntindex[Template]{}

The global imperative to mitigate climate change has placed the transportation sector at the center of sustainability debates, as it remains one of the largest contributors to greenhouse gas (GHG) emissions worldwide, it accounts for roughly 23\%\ of total greenhouse gas emissions \parencite{IPCC2022Transport}. Within this sector, Urban Freight Transportation (UFT) plays a critical yet often underestimated role. Urban freight activities are essential to the functioning of modern cities, ensuring that goods are efficiently delivered to businesses and consumers. At the same time, they significantly contribute to traffic congestion, local air pollution, and carbon emissions, particularly in densely populated urban areas \parencite{HAO201594}. Consequently, urban freight systems are increasingly subject to international decarbonization targets and stringent local environmental regulations.

In response, public authorities across the world have adopted a wide range of policy measures to manage and decarbonize urban freight transport. A comprehensive international review by \textcite{HolguinVeras2015a} \parencite{HolguinVeras2015b}, shows that cities across Europe, North America, Asia, and Latin America have implemented initiatives such as access restrictions, low-emission zones, vehicle-related regulations, traffic management schemes, and infrastructure adaptations. Evidence from 56 cities in 32 countries indicates that regulatory and supply-side approaches dominate current practice, with Europe and Asia leading the implementation of vehicle-related strategies, such as emission standards and low-emission zones, while North America relies more heavily on infrastructure and operational management measures.

Although these policies have been effective in reducing emissions, congestion, and other negative externalities, they frequently impose additional costs on freight operators. In particular, vehicle-related restrictions and low-emission zones are associated with increased delivery costs and accelerated fleet renewal, highlighting a fundamental trade-off between environmental objectives and operational profitability. As a result, urban freight policies increasingly transfer financial and investment risk to private fleet operators, who must adapt their vehicle portfolios under regulatory pressure while facing uncertain economic returns.

Despite the breadth and intensity of these policy interventions, their translation into large-scale fleet transformation has been uneven. A systematic literature review by \textcite{KervallPalsson2022}, covering 93 peer-reviewed studies, identifies a wide range of barriers that hinder the transformation of urban freight systems, including economic constraints, technological uncertainty, lack of charging infrastructure, and policy misalignment. Among these, economic barriers and the so-called “first-mover disadvantage” are particularly critical for fleet operators, as early adopters of electric vehicles face high upfront investment costs, uncertain operational benefits, and volatile energy prices. As a consequence, fleet replacement decisions are often postponed or limited to incremental changes, despite strong regulatory pressure.

This combination of regulatory ambition and systemic barriers transforms fleet replacement into a complex decision-making problem under uncertainty. While electric vehicles offer clear environmental benefits and the potential for lower operating costs, they are associated with high capital expenditures and significant technological uncertainty. Fleet managers must make high-stakes investment decisions in a volatile environment, where fuel prices, electricity costs, and vehicle residual values fluctuate unpredictably. The central challenge, therefore, is not merely environmental compliance, but determining an optimal fleet composition that balances conventional and alternative vehicle technologies while meeting sustainability targets without compromising financial viability.

\section{Previous approach and objectives}
\label{sec:Previous_approach_and_objectives}

To support investment decisions under uncertainty in fleet replacement problems, the financial literature offers well-established quantitative tools, most notably Modern Portfolio Theory (MPT), originally introduced by \textcite{Markowitz1952}. MPT provides a framework for balancing expected performance and risk by treating different assets as components of a portfolio with distinct cost and risk characteristics. This framework has been successfully adapted to the context of urban freight transportation by \textcite{Ahani2016}, who modeled vehicle technologies as portfolio assets and demonstrated that a diversified fleet composition can hedge against uncertainty in fuel prices and operating costs.

In the formulation proposed by \textcite{Ahani2016}, the fleet replacement problem is modeled as a multi-objective optimization problem, with the objectives of minimizing the Expected Total Cost (ETC) and minimizing the Variance of the total cost. The expected cost captures average economic performance, while variance is used as a measure of risk, reflecting cost volatility arising from uncertain market conditions. This mean–variance formulation allows decision-makers to explicitly analyze the trade-off between cost efficiency and risk exposure, providing a structured basis for fleet composition decisions under uncertainty.

Despite its relevance and practical applicability, existing applications of the mean–variance framework in urban fleet optimization are often based on simplified representations of real-world fleets. In particular, many models focus on a limited number of vehicle technologies or assume homogeneous vehicle characteristics, which restricts their ability to capture the diversity observed in actual urban freight fleets. Moreover, the increasing availability of alternative vehicle technologies, such as electric, diesel, and hybrid vehicles with different sizes and operational profiles, calls for more comprehensive modeling approaches that can reflect this heterogeneity while preserving analytical tractability.


Building on the framework of \textcite{Ahani2016}, this thesis extends the mean–variance approach by developing a more detailed fleet replacement optimization model that accommodates multiple vehicle technologies and characteristics. The resulting problem is formulated as a multi-objective mixed-integer linear programming (MILP) model. To solve this problem, the \texorpdfstring{$\varepsilon$}{ε}-constraint method is employed, whereby one objective is optimized while the other is constrained within predefined bounds. By systematically varying the $\varepsilon$ parameter, the efficient frontier of optimal fleet compositions can be generated, enabling a clear visualization and analysis of the trade-offs between expected cost and risk.

The primary objective of this thesis is therefore to develop and analyze a mean–variance-based optimization framework for urban fleet replacement decisions under uncertainty. Specifically, this work aims to:

\begin{itemize}
    \item Formulate a multi-objective optimization model that minimizes both Expected Total Cost and cost Variance in urban fleet replacement decisions;
    \item Extend existing models by incorporating multiple vehicle technologies and fleet characteristics;
    \item Apply the $\varepsilon$-constraint method to solve the resulting mixed-integer linear problem and generate the efficient frontier;
    \item Analyze the trade-offs between cost and risk under different fleet compositions and market conditions.
\end{itemize}

In this context, the proposed framework is intended to support both private and public decision-makers by providing a transparent and quantitative tool for evaluating fleet replacement strategies that balance economic performance and risk in the transition toward more sustainable urban freight systems.

\section{Research Questions}
\label{sec:Research_Questions}

To address the identified gaps and guide the development of the proposed optimization framework, this thesis aims to answer the following core research questions:

\begin{itemize}
    \item \textbf{RQ1:} How does the trade-off between Expected Total Cost and cost Variance influence the optimal composition of an urban freight fleet under uncertainty?
    
    \item \textbf{RQ2:} How do different levels of risk aversion, represented through a mean–variance framework, affect fleet replacement decisions and the resulting efficient frontier?
    
    \item \textbf{RQ3:} How does the trade-off between Expected Total Cost and cost Variance,
    as represented by the efficient frontier, influence the optimal transition from conventional
    vehicle technologies to electric and hybrid alternatives in urban freight fleets?
    
    \item \textbf{RQ4:} How sensitive are the Pareto-optimal fleet configurations to variations
    in key market and policy parameters (energy prices, vehicle acquisition costs, and
    carbon pricing), and which parameter thresholds trigger significant shifts in optimal fleet
    composition?
\end{itemize}

%% ---------------------------- -------------------------------- -----------------------------
%% ---------------------------- -------Document Structure------- -----------------------------
%% ---------------------------- -------------------------------- -----------------------------
\section{Document Structure}
\label{sec:Document_Structure}

The remainder of this dissertation is organized as follows.

Chapter 2 presents a comprehensive literature review on sustainable urban freight transportation, fleet replacement optimization models, and the application of portfolio theory to transportation systems. Special attention is given to mean–variance formulations and risk modeling approaches relevant to fleet investment decisions under uncertainty.

Chapter 3 describes the case study considered in this work. It introduces the characteristics of the urban freight fleet, the operational context, and the data assumptions used to represent vehicle technologies, cost components, and uncertainty sources.

Chapter 4 presents the proposed optimization framework. This chapter details the mathematical formulation of the multi-objective fleet replacement model, including decision variables, objective functions, and constraints. The solution approach based on the $\varepsilon$-constraint method is also described, along with the procedure used to generate the efficient frontier.

Chapter 5 outlines the tasks to be carried out for the remainder of this dissertation, along with the planned schedule for their completion.

\newpage