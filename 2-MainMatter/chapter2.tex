%!TEX root = ../template.tex
%%%%%%%%%%%%%%%%%%%%%%%%%%%%%%%%%%%%%%%%%%%%%%%%%%%%%%%%%%%%%%%%%%%%
%% chapter2.tex
%% NOVA thesis document file
%%
%% Chapter with the template manual
%%%%%%%%%%%%%%%%%%%%%%%%%%%%%%%%%%%%%%%%%%%%%%%%%%%%%%%%%%%%%%%%%%%%

\typeout{NT FILE chapter2.tex}%

\chapter{Literature review}
\label{cha:Literature_review}

This chapter reviews the academic literature relevant to urban fleet replacement decisions under uncertainty. The review is structured to progressively narrow the focus from the broader context of sustainable urban logistics to the specific methodological tools used in this thesis. First, research on sustainable urban freight transportation and the transition toward electric vehicles is discussed. Next, classical and modern formulations of the fleet replacement problem are reviewed. The chapter then introduces portfolio-based approaches in transportation, followed by a detailed discussion of total cost formulations and risk modeling using a mean-variance framework. The chapter concludes by identifying gaps in the existing literature that motivate the optimization framework developed in Chapter~\ref{cha:Proposed_Approach}.

\glsresetall

\section{Sustainable Urban Logistics and the EV Transition}
\label{sec:Sustainable_Urban_Logistics_and_the_EV_Transition}

Driven by e-commerce and urbanization, last-mile delivery has grown substantially and it has intensified the challenges of urban freight distribution. As noted by \textcite{Ahani2016}, many studies have been published with the goal of dealing GHG, noise emission and congestion arising from the freight movements inside urban populated areas (\cite{TipagornwongFigliozzi2014, RUSSO201261, Dablanc2007,OECD2003AnnualReport,PelletierJabaliLaporte2016,Barter2012}). Urban logistics is a primary source of congestion and local pollution. Consequently, regulatory bodies are implementing stricter measures, such as Low Emission Zones (LEZ) and carbon taxes, forcing operators to take a second and think carefully about their fleet composition. \par

Electric Vehicles (EVs) have emerged as the leading alternative to Internal Combustion Engine Vehicles (ICEVs). Studies indicate that while EVs have higher acquisition costs, their operational costs (energy and maintenance) are significantly lower. However, the transition is not trivial. Authors like \textcite{Nina2010} and \textcite{Pelletier2019EBFTP} highlight that the "Total Cost of Ownership" (TCO) depends heavily on usage intensity, charging infrastructure availability, and battery lifespan uncertainty. Therefore, the decision to switch is not merely operational but a strategic investment under uncertainty. \par

Furthermore, the economic environment for fleet operators is becoming increasingly volatile. According to the \textcite{WorldBank2025CMO} \citetitle{WorldBank2025CMO}, energy prices are subject to significant downside and upside risks driven by geopolitical tensions and supply chain disruptions. This report highlights that while supply conditions might stabilize some commodity prices, the risks of extreme fluctuations remain. For a fleet manager, this reinforces that relying on static cost averages is dangerous; a robust model must account for these potential extreme price swings (tail risks).

\section{Fleet Replacement Problem}
\label{sec:Fleet_Replacement_Problem}

Fleet replacement is a classical decision problem in operations research, traditionally formulated as a
trade-off between increasing operating and maintenance costs of aging assets and the net cost of
replacement, normally taking into account a budget, capacity, or service constraints. Early formulations typically relied on
deterministic assumptions, treating future fuel prices, vehicle costs, and demand as known parameters \parencite{Emiliano2020}.
In another study, \textcite{FengFigliozzi2013} present a deterministic integer linear programming model that integrates the vehicle replacement problem with the fleet size and mix problem for commercial fleets, enabling a comparison between electric vehicles (EVs) and diesel vehicles (DVs).
While such models can yield useful baseline insights, they may be fragile in practice when exposed to
volatility in markets, policy, and technology. Therefore, when considering real-world volatility, deterministic models are insufficient. \par

To address this limitation, recent contributions increasingly adopt stochastic and robust optimization
approaches. For example, \textcite{Onyshchenko2024} discuss integrated models for complex transport systems, emphasizing the need to account for variable parameters. Similarly, \textcite{Ahani2023UFT} introduced regulatory constraints into replacement models, showing that policy uncertainty significantly impacts optimal replacement timing. The consensus is clear: models must account for randomness in input parameters (fuel prices, demand, technology costs) to provide robust solutions. In these models, uncertainty is represented via scenarios or probability distributions and
embedded directly into the decision-making process, enabling the derivation of strategies that are less
sensitive to adverse realizations of uncertain inputs. Within sustainable transportation contexts,
regulatory constraints and policy mechanisms have become central modeling components, reflecting
the growing influence of decarbonization policies on fleet investment decisions.

A representative example is the fleet replacement model of \textcite{Xiao2023}, who formulate a mixed-integer
programming framework under a cap-and-trade system while explicitly incorporating uncertainty in
government subsidies over the planning horizon. The study considers multiple vehicle technologies and alternative business models such as battery
leasing. This demonstrates that policy uncertainty should not be ignored because it can be a
quantitative driver of optimal replacement timing and fleet composition.

In parallel, other research focuses on uncertainty that emerges from operational conditions rather than
policy or market parameters. \textcite{Malladi2020} propose a two-stage stochastic program for fleet size and
mix decisions in urban logistics with uncertain customer requests and ambient temperature, and it
highlights that temperature-dependent auxiliary energy consumption can substantially affect EV range
and operational feasibility. Such models emphasize the tight coupling between strategic fleet decisions
and operational performance, particularly when EV energy consumption is sensitive to environmental
factors.

Finally, \textcite{Kaszubowski2019} proposes a three-tier qualitative method to evaluate urban freight models, emphasizing the need for tools that align strategic objectives with operational data. Similarly, a systematic review by \textcite{AlvarezGallo2023} highlights the extensive use of Multi-Criteria Decision Analysis (MCDA) to balance conflicting goals like sustainability and cost. However, while MCDA is excellent for ranking strategic alternatives, it often lacks the mathematical capacity to quantify financial risk distributions explicitly. This thesis moves beyond qualitative ranking to quantitative portfolio optimization, specifically addressing the financial risk of asset replacement described by \textcite{Ahani2016}.

\section{The Portfolio Approach in Transportation}
\label{sub:The_Portfolio_Approach_in_Transportation}

In order to handle these uncertainties, researchers have used and adapted the Modern Portfolio Theory (MPT). The core premise of the adapted MPT is that a fleet of vehicles can be treated as a portfolio of assets.

\begin{itemize}
    \item Assets: Vehicle technologies (Diesel, Electric, Hybrid).
    \item Return: The cost savings relative to a baseline.
    \item Risk: The volatility of these savings.
\end{itemize}

The work by \textcite{Ahani2016} pioneered this approach for urban fleets. They formulated a Multi-Objective Optimization model where the objective was to minimize the Expected Total Cost (ETC) and minimize the Variance of the Total Cost. Their results demonstrated that a diversified fleet (mixing ICEVs and EVs) provides a hedge against fuel price spikes, similar to how a diversified stock portfolio protects against market crashes. This methodology provides the mathematical foundation for this thesis.

While the portfolio perspective is conceptually appealing and mathematically well-founded, many existing
implementations focus on simplified fleet representations, for example, a limited number of technologies or
homogeneous vehicle characteristics. Therefore, there is space for extending the framework
to accommodate multiple vehicle technologies and characteristics, aiming to better reflect the
heterogeneity of real urban freight fleets while preserving analytical tractability.

\section{Total Cost Formulation in Fleet Replacement Models}
\label{sub:Total_Cost_Formulation_in_Fleet_Replacement_Models}

In fleet replacement optimization problems, the definition of total cost plays a central role in evaluating alternative fleet compositions and investment strategies. Following the framework proposed by \textcite{Ahani2016}, the total cost of a fleet over the planning horizon is typically composed of several cost components, including vehicle acquisition costs, energy costs, maintenance and operation expenses, and residual values at the end of the vehicle life cycle.

Vehicle acquisition costs represent the upfront capital expenditure associated with purchasing new vehicles of different technologies. Energy costs include fuel or electricity consumption and depend on vehicle efficiency, usage intensity, and energy prices, which may vary over time and across scenarios. Maintenance and operating costs capture routine servicing, repairs, and other operational expenses that are influenced by vehicle technology and age. In some formulations, residual values or replacement costs are also considered to account for the remaining value of vehicles at the end of the planning horizon.

When uncertainty is present, some components of the total cost, most notably energy prices and operating expenses, are treated as stochastic variables. In this case, the total cost becomes a random variable, whose expected value and variance can be computed across a finite set of scenarios. This representation enables the integration of total cost modeling with portfolio-based optimization approaches, where Expected Total Cost is minimized while controlling for cost variability.

The fleet replacement problem is formulated over a discrete planning horizon and a finite set of vehicle type. The main sets and indices used in the model are defined as follows:

\begin{itemize}
    \item $t \in \mathcal{T} = \{0, \dots, T\}$: index of time periods.
    \item $i \in \mathcal{I}_k = \{0, \dots, A_k\}$: index of vehicle age for type $k$.
    \item $k \in \mathcal{K} = \{1, \dots, K\}$: index of vehicle type.
\end{itemize}

The following decision variables are used in the optimization model:

\begin{itemize}
    \item $V_{i,t,k}$: number of vehicles of technology $k$ and age $i$ operated in period $t$.
    \item $P_{t,k}$: number of new vehicles of technology $k$ purchased at the beginning of period $t$.
    \item $S_{i,t,k}$: number of vehicles of technology $k$ and age $i$ salvaged at the end of period $t$.
\end{itemize}

All decision variables are non-negative integers, reflecting the discrete nature of fleet composition and replacement decisions.
The total cost formulation consists of the following components.

%% ------------------------- TOTAL ENERGY COST -------------------------
Energy Cost: The total energy cost incurred by operating the fleet is given by:
\begin{equation}
EC = \sum_{i=0}^{A_k-1}\sum_{t=0}^{T-1}\sum_{k=1}^{K}
f_{i,t,k}\,u_{i,k}\,V_{i,t,k}\,(1+dr)^{-t},
\label{eq:EC}
\end{equation}
where $f_{i,t,k}$ represents the unit energy cost per kilometer (\(\text{€}/\mathrm{km}\)) for a vehicle of age $i$ and technology $k$ in period $t$, $u_{i,k}$ is the annual utilization and $dr$ denotes the discount rate.

%% ----------------------- OPEX -------------------------
Maintenance Cost: Maintenance and operating costs are modeled as:
\begin{equation}
MC = \sum_{i=0}^{A_k-1}\sum_{t=0}^{T-1}\sum_{k=1}^{K}
m_{i,t,k}\,u_{i,k}\,V_{i,t,k}\,(1+dr)^{-t},
\label{eq:MC}
\end{equation}
where $m_{i,t,k}$ denotes the maintenance cost per kilometer of vehicle k of an age i during period t per (€/km).

%% ----------------------- CAPEX -------------------------
Capital Investment Cost: Associated with acquiring new vehicles is expressed as:
\begin{equation}
CIC = \sum_{t=0}^{T-1}\sum_{k=1}^{K}
v_{k,t}\,P_{t,k}\,(1+dr)^{-t},
\label{eq:CIC}
\end{equation}
where $v_{k,t}$ is the purchase cost (€) per unit of a vehicle of technology $k$ in period $t$.

%% ----------------------- EMISSION COST -------------------------
Emission Related Cost:
\begin{equation}
ERC = \sum_{i=0}^{A_k-1}\sum_{t=0}^{T-1}\sum_{k=1}^{K}
e_{i,k}\,u_{i,k}\,V_{i,t,k}\,(1+dr)^{-t},
\label{eq:R}
\end{equation}
where $e_{i,k}$ represents CO$_2$ emission cost (€/km) of vehicle of age $i$, type $k$.

%% ----------------------- SALVAGE REVENUE -------------------------
Salvage Revenue: Obtained from retiring vehicles at the end of their service life is given by:
\begin{equation}
SR = \sum_{i=1}^{A_k}\sum_{t=0}^{T}\sum_{k=1}^{K}
s_{i,k}\,S_{i,t,k}\,(1+dr)^{-t},
\label{eq:SR}
\end{equation}
where $s_{i,k}$ denotes the salvage revenue (€) of a vehicle of age $i$ and technology $k$.

%% ----------------------- TOTAL COST -------------------------
Total Cost: By combining all cost components of the fleet over the planning horizon can be expressed as:

\begin{equation}
\begin{aligned}
TC =\;& \sum_{t=0}^{T-1}\sum_{k=1}^{K} v_{k,t}\,P_{t,k}\,(1+dr)^{-t} - \sum_{i=1}^{A_k}\sum_{t=0}^{T}\sum_{k=1}^{K} s_{i,k}\,S_{i,t,k}\,(1+dr)^{-t} \\
&+ \sum_{i=0}^{A_k-1}\sum_{t=0}^{T-1}\sum_{k=1}^{K}
\bigl(f_{i,t,k}+m_{i,t,k}+e_{i,k}\bigr)\,u_{i,k}\,V_{i,t,k}\,(1+dr)^{-t}.
\end{aligned}
\label{eq:TC1}
\end{equation}

When uncertainty is present in parameters such as energy prices, maintenance costs, or vehicle prices, the total cost $TC$ becomes a random variable. This representation allows the Expected Total Cost and the variance of total cost to be computed across a finite set of scenarios, providing the basis for mean–variance optimization in fleet replacement decisions.

By adopting the total cost formulation proposed by \textcite{Ahani2016}, this thesis ensures consistency with established models in the literature while providing a solid foundation for extending the analysis to more detailed fleet representations and multi-objective optimization frameworks.

\section{Risk Modeling in Mean–Variance Fleet Optimization}
\label{sub:Risk_Modeling_in_Mean–Variance_Fleet_Optimization}

Based on the total cost formulation described in the previous section, uncertainty in urban fleet replacement can be represented through the variability of total cost outcomes across time and operating conditions. In practice, fleet investment decisions are exposed to market and technology uncertainty (energy prices, acquisition costs, residual values), operational uncertainty (demand and route requirements that affect utilization), and policy uncertainty (incentives, emissions pricing, and regulatory constraints). These uncertainties make the total discounted cost of operating and renewing a fleet a stochastic quantity rather than a deterministic value. Consequently, decisions which involve risk requires methods that do not rely on average cost assumptions, but instead captures the degree of cost volatility faced by decision-makers.

In this thesis, risk is modeled using a mean–variance approach. Expected Total Cost (ETC) captures average economic performance, while the variance of total cost captures costvolatility under uncertainty. This provides a transparent mechanism to represent differentrisk attitudes and to identify fleet compositions that balance cost efficiency with financial stability.

\subsection{Cost Uncertainty in Urban Fleet Replacement}
\label{sub:Cost_Uncertainty_in_Urban_Fleet_Replacement}

Urban freight fleet replacement decisions are characterized by significant uncertainty arising from multiple cost components. Key sources of uncertainty include fuel and electricity prices, vehicle acquisition costs, maintenance and repair expenses, and residual values at the end of a vehicle’s service life. These cost elements are influenced by external factors such as energy market volatility, technological progress, regulatory changes, and macroeconomic conditions, making long-term cost estimation inherently uncertain.

In this context, relying solely on deterministic or average cost values can lead to misleading conclusions and suboptimal investment decisions. Instead, fleet replacement models must explicitly account for uncertainty in order to provide solutions that are robust to fluctuations in key parameters. This has motivated the use of stochastic optimization approaches in the literature, where uncertainty is represented through scenarios or probability distributions and incorporated directly into the decision-making process.

\subsection{Variance as a Measure of Risk}
\label{sub:Variance_as_a_Measure_of_Risk}

Within stochastic optimization and portfolio-based approaches, risk is commonly quantified through the variance of total cost. Variance measures the dispersion of cost outcomes around their expected value and captures the degree of cost volatility faced by decision-makers. In the context of urban freight fleets, a higher variance indicates greater exposure to unpredictable cost fluctuations, which may challenge budget planning and financial stability.

The use of variance as a risk measure is well established in both financial economics and transportation research. Its mathematical properties make it particularly attractive for optimization models, as it is intuitive, widely understood, and compatible with quadratic or linearized formulations. Moreover, variance provides a transparent way to represent risk preferences by allowing decision-makers to balance expected cost minimization against acceptable levels of cost variability.

In fleet replacement problems, variance reflects the combined uncertainty associated with different vehicle technologies and energy sources. By diversifying the fleet composition across technologies with distinct cost structures and uncertainty profiles, operators can reduce overall cost volatility, in line with the principles of Modern Portfolio Theory.

While variance does not explicitly capture tail risk, its analytical tractability and intuitive interpretation make it particularly suitable for strategic fleet planning problems. Variance-based risk measures are widely used in both financial economics and transportation research and can be naturally integrated into optimization models.

\subsection{Mean–Variance Formulation in Fleet Replacement Models}
\label{sub:Mean–Variance_Formulation_in_Fleet_Replacement_Models}
Building on the principles of Modern Portfolio Theory, the mean–variance approach has been adapted to urban fleet replacement problems by modeling vehicle technologies as portfolio assets with uncertain costs. In this framework, the Expected Total Cost (ETC) represents the mean performance criterion, while the variance of total cost captures the associated risk. The fleet replacement problem is thus formulated as a multi-objective optimization problem that seeks to minimize both ETC and cost variance.

A seminal application of this approach in urban freight transportation is provided by \textcite{Ahani2016}, who demonstrated that a diversified fleet composed of conventional and alternative fuel vehicles can hedge against uncertainty in fuel prices and operating costs. Their results showed that the mean–variance framework allows decision-makers to explore the trade-offs between cost efficiency and risk exposure, leading to a set of Pareto-optimal fleet configurations rather than a single optimal solution.

From a modeling perspective, the mean–variance formulation is particularly suitable for mixed-integer linear programming (MILP) extensions of fleet replacement problems. When cost uncertainty is represented through a finite set of scenarios, the expected cost and variance can be computed explicitly and incorporated into the optimization model. This enables the systematic exploration of cost–risk trade-offs using established multi-objective solution techniques, such as the $\varepsilon$-constraint method, while maintaining computational tractability.

Overall, the mean–variance framework provides a coherent and well-founded approach for modeling risk in urban fleet replacement decisions. It allows the explicit representation of uncertainty, supports diversification across vehicle technologies, and offers decision-makers a transparent tool to balance economic performance and risk in the transition toward more sustainable urban freight systems.
