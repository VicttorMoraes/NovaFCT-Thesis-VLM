%!TEX root = ../template.tex
%%%%%%%%%%%%%%%%%%%%%%%%%%%%%%%%%%%%%%%%%%%%%%%%%%%%%%%%%%%%%%%%%%%%
%% chapter2.tex
%% NOVA thesis document file
%%
%% Chapter with the template manual
%%%%%%%%%%%%%%%%%%%%%%%%%%%%%%%%%%%%%%%%%%%%%%%%%%%%%%%%%%%%%%%%%%%%

\typeout{NT FILE chapter2.tex}%

\chapter{Literature review}
\label{cha:literature_review}

This chapter reviews the academic literature relevant to urban fleet replacement decisions under uncertainty. The review is structured to progressively narrow the focus from the broader context of sustainable urban logistics to the specific methodological tools used in this thesis. First, research on sustainable urban freight transportation and the transition toward electric vehicles is discussed. Next, classical and modern formulations of the fleet replacement problem are reviewed. The chapter then introduces portfolio-based approaches in transportation, followed by a detailed discussion of total cost formulations and risk modeling using a mean-variance framework. The chapter concludes by identifying gaps in the existing literature that motivate the optimization framework developed in Chapter~\ref{cha:Proposed_Approach}.

\glsresetall

\section{Sustainable Urban Logistics and the EV Transition}
\label{sec:Sustainable_Urban_Logistics_and_the_EV_Transition}

Driven by e-commerce and urbanization, last mile delivery has grown substantially and it has intensified the challenges of urban freight distribution. As noted by \textcite{Ahani2016}, many studies have been published with the goal of dealing GHG, noise emission and congestion arising from the freight movements inside urban populated areas (\textcite{TipagornwongFigliozzi2014,PelletierJabaliLaporte2016}). Urban logistics is a primary source of congestion and local pollution. Consequently, regulatory bodies have implemented stricter measures, such as Low Emission Zones (LEZ) and carbon taxes, forcing operators to take a second and think carefully about their fleet composition (\textcite{BroaddusBrowneAllen2015,Watkiss2003LondonLEZ}). \par

Electric Vehicles (EVs) have emerged as the leading alternative to Internal Combustion Engine Vehicles (ICEVs). Studies indicate that while EVs have higher acquisition costs, their operational costs (energy and maintenance) are significantly lower. However, the transition is not trivial. Authors like \textcite{Nina2010} and \textcite{Pelletier2019EBFTP} highlight that the "Total Cost of Ownership" (TCO) depends heavily on usage intensity, charging infrastructure availability, and battery lifespan uncertainty. Therefore, the decision to switch is not merely operational but a strategic investment under uncertainty. \par

Furthermore, the economic environment for fleet operators is becoming increasingly volatile. According to the \textcite{WorldBank2025CMO} \citetitle{WorldBank2025CMO}, energy prices are subject to significant downside and upside risks driven by geopolitical tensions and supply chain disruptions. This report highlights that while supply conditions might stabilize some commodity prices, the risks of extreme fluctuations remain. For a fleet manager, this reinforces that relying on static cost averages is dangerous; a robust model must account for these potential extreme price swings (tail risks).

\subsection{Fleet Replacement Problem}
\label{sec:Fleet_Replacement_Problem}

Fleet replacement is a classical decision problem in operations research, traditionally
formulated as a trade-off between increasing operating and maintenance costs of aging
assets and the net cost of replacement, typically subject to budget, capacity, and service
constraints. Early formulations often relied on deterministic assumptions, treating future
fuel prices, vehicle acquisition costs, and demand as known parameters. While such models
can yield useful baseline insights, they may be fragile in practice when exposed to volatility
in markets, policy, and technology, consequently, purely deterministic models can be
insufficient for real-world fleet investment planning.

Beyond mean-variance portfolio formulations, the fleet replacement problem has been
addressed through a variety of optimization approaches. A common baseline is deterministic
or mixed-integer programming (MIP) modeling, where replacement timing and fleet mix are
optimized under budget and operational constraints, often incorporating environmental costs
or emissions policies. For instance, \textcite{FengFigliozzi2013} propose an integer programming
framework to analyze the economic competitiveness of electric versus conventional commercial
vehicles across utilization and price scenarios. Similarly, \textcite{FigliozziBoudartFeng2011}
develop a vehicle replacement model that explicitly captures trade-offs between economic and
environmental objectives, including taxes and incentives. In the public transport context,
\textcite{Pelletier2019EBFTP} formulate the electric bus fleet transition problem, integrating
replacement planning with charging infrastructure investment decisions and electrification targets.

More recent studies extend these replacement models by embedding uncertainty and policy
mechanisms directly into the decision process. For example, \textcite{Xiao2023} develop a fleet
replacement optimization model under a cap-and-trade system while accounting for uncertainty
in government subsidies and alternative business models (e.g., leasing). In parallel, other work
highlights uncertainty arising from operational conditions: \textcite{Malladi2020} formulate a
two-stage stochastic program for fleet size and mix decisions under uncertain customer requests
and temperature-dependent energy consumption, showing that operational variability can
materially affect optimal fleet composition and the feasibility of electrification.

Finally, complementary strands of the literature emphasize evaluation and decision-support
methods that do not rely exclusively on optimization-based replacement models. For example,
\textcite{Kaszubowski2019} proposes a qualitative multi-tier method to evaluate urban freight
models, while systematic reviews like \textcite{AlvarezGallo2023} document the use
of Multi-Criteria Decision Analysis (MCDA) to balance sustainability, cost, and service criteria.
However, while MCDA is effective for ranking strategic alternatives, it typically does not
quantify cost-risk trade-offs under uncertainty via an explicit efficient frontier. This motivates
the adoption of portfolio-based approaches, which directly characterize the trade-off between
expected cost and cost variability across scenarios.


\section{The Portfolio Approach in Transportation}
\label{sub:The_Portfolio_Approach_in_Transportation}

In order to handle these uncertainties, researchers have used and adapted the Modern Portfolio Theory (MPT). The core premise of the adapted MPT is that a fleet of vehicles can be treated as a portfolio of assets.

\begin{itemize}
    \item Assets: Vehicle technologies (Diesel, Electric, Hybrid).
    \item Return: The cost savings relative to a baseline.
    \item Risk: The volatility of these savings.
\end{itemize}

The work by \textcite{Ahani2016} pioneered this approach for urban fleets. They formulated a Multi-Objective Optimization model where the objective was to minimize the Expected Total Cost (ETC) and minimize the Variance of the Total Cost. Their results demonstrated that a diversified fleet (mixing ICEVs and EVs) provides a hedge against fuel price spikes, similar to how a diversified stock portfolio protects against market crashes. This methodology provides the mathematical foundation for this thesis.

While the portfolio perspective is conceptually appealing and mathematically well-founded, many existing
implementations focus on simplified fleet representations, for example, a limited number of technologies or
homogeneous vehicle characteristics. Therefore, there is space for extending the framework
to accommodate multiple vehicle technologies and characteristics, aiming to better reflect the
heterogeneity of real urban freight fleets while preserving analytical tractability.

\section{Total Cost in Fleet Replacement Models}
\label{sub:Total_Cost_Formulation_in_Fleet_Replacement_Models}

In fleet replacement optimization problems, the definition of total cost plays a central role in evaluating alternative fleet compositions and investment strategies. Following the framework proposed by \textcite{Ahani2016}, the total cost of a fleet over the planning period is typically composed of several cost components, including vehicle acquisition costs, energy costs, maintenance and operation expenses, and residual values at the end of the vehicle life cycle.

Vehicle acquisition costs represent the upfront capital expenditure associated with purchasing new vehicles of different technologies. Energy costs include fuel or electricity consumption and depend on vehicle efficiency, usage intensity, and energy prices, which may vary over time and across scenarios. Maintenance and operating costs capture routine servicing, repairs, and other operational expenses that are influenced by vehicle technology and age. In some formulations, residual values or replacement costs are also considered to account for the remaining value of vehicles at the end of the planning period.

When uncertainty is present, some components of the total cost, most notably energy prices and operating expenses, are treated as stochastic variables. In this case, the total cost becomes a random variable, whose expected value and variance can be computed across a finite set of scenarios. This representation enables the integration of total cost modeling with portfolio-based optimization approaches, where Expected Total Cost is minimized while controlling for cost variability.

\section{Risk Modeling in Fleet Optimization}
\label{sub:Risk_Modeling_in_Mean–Variance_Fleet_Optimization}

Based on the total cost formulation described in the previous section, uncertainty in urban fleet replacement can be represented by the dispersion of total cost outcomes across alternative future states. In practice, long-term fleet investment decisions are exposed to market and technology uncertainty (fuel and electricity prices, acquisition costs, and residual values), operational uncertainty (demand and routing requirements that affect utilization), and policy uncertainty (incentives, emissions pricing, and access restrictions). These drivers make the total discounted cost of renewing and operating a fleet inherently stochastic rather than deterministic, so relying on point estimates or average values can lead to misleading conclusions and suboptimal investments. Consequently, the literature commonly models uncertainty through scenarios or probability distributions and embeds it directly into the optimization problem to evaluate candidate fleet strategies under a range of plausible conditions \parencite{MALLADI2022102554,Ahani2023UFT}. Once uncertainty is represented, the remaining modeling choice is how to quantify and penalize risk across scenarios.

To demonstrate that risk can be modeled in multiple ways, it is useful to distinguish between (i) deviation-based measures that capture dispersion using linear constructs and (ii) tail-focused measures that explicitly represent exposure to extreme adverse outcomes. These alternatives are well established in risk-aware optimization and can be integrated into scenario-based fleet replacement models with different computational implications \parencite{MANSINI2014518}.

\subsection{Variance as a Measure of Risk}
\label{sub:Variance_as_a_Measure_of_Risk}

Within stochastic optimization and portfolio-based approaches, risk is commonly quantified through the variance of total cost. Variance measures the dispersion of cost outcomes around their expected value and captures the degree of cost volatility faced by decision-makers. In the context of urban freight fleets, a higher variance indicates greater exposure to unpredictable cost fluctuations, which may challenge budget planning and financial stability.

The use of variance as a risk measure is well established in both financial economics and transportation research. Its mathematical properties make it particularly attractive for optimization models, as it is intuitive, widely understood, and compatible with quadratic or linearized formulations. Moreover, variance provides a transparent way to represent risk preferences by allowing decision-makers to balance expected cost minimization against acceptable levels of cost variability.

In fleet replacement problems, variance reflects the combined uncertainty associated with different vehicle technologies and energy sources. By diversifying the fleet composition across technologies with distinct cost structures and uncertainty profiles, operators can reduce overall cost volatility, in line with the principles of Modern Portfolio Theory.

While variance does not explicitly capture tail risk, its analytical tractability and intuitive interpretation make it particularly suitable for strategic fleet planning problems. Variance-based risk measures are widely used in both financial economics and transportation research and can be naturally integrated into optimization models.

\subsection{Mean Absolute Deviation as a Measure of Risk}
\label{sub:mad}

A known linear alternative to variance is the Mean Absolute Deviation (MAD), which measures the expected absolute deviation of scenario outcomes around a central tendency, often the mean. Under a discrete scenario set, MAD can be linearized using auxiliary variables, leading to MILP-compatible formulations while preserving an interpretable dispersion based notion of risk \parencite{MANSINI2014518}.

\subsection{Conditional Value-at-Risk as a Measure of Risk}
\label{sub:cvar}

Tail-risk measures focus on the worst outcomes rather than overall dispersion. Conditional Value-at-Risk (CVaR), also known as Expected Shortfall, captures the expected cost in the worst $(1-\alpha)$ fraction of scenarios. In scenario-based optimization, CVaR admits an equivalent formulation using auxiliary variables and linear constraints, which makes it attractive for risk-aware planning models \parencite{RockafellarUryasev2000}. Importantly, CVaR has already been used directly in sustainable fleet replacement contexts to represent cost risk under uncertain fuel and carbon-related parameters, demonstrating its relevance beyond purely financial applications \parencite{ANSARIPOOR2014701}.

\subsection{Entropic Value-at-Risk as a Measure of Risk}
\label{sub:evar}

Recent research has also proposed non-linear coherent measures such as Entropic Value-at-Risk (EVaR), which upper bounds tail risk through exponential (Chernoff) bounds and can yield tractable convex formulations in certain stochastic optimization settings \parencite{AhmadiJavid2011}. EVaR provides an example of a non-linear risk modeling paradigm and helps position variance-based risk within a broader methodological landscape.

\section{Mean–Variance Formulation in Fleet Replacement Models}
\label{sec:Mean–Variance_Formulation_in_Fleet_Replacement_Models}
Building on the principles of Modern Portfolio Theory, the mean–variance approach has been adapted to urban fleet replacement problems by modeling vehicle technologies as portfolio assets with uncertain costs. In this framework, the Expected Total Cost (ETC) represents the mean performance criterion, while the variance of total cost captures the associated risk. The fleet replacement problem is thus formulated as a multi-objective optimization problem that seeks to minimize both ETC and cost variance.

A seminal application of this approach in urban freight transportation is provided by \textcite{Ahani2016}, who demonstrated that a diversified fleet composed of conventional and alternative fuel vehicles can hedge against uncertainty in fuel prices and operating costs. Their results showed that the mean–variance framework allows decision-makers to explore the trade-offs between cost efficiency and risk exposure, leading to a set of Pareto-optimal fleet configurations rather than a single optimal solution.

From a modeling perspective, the mean-variance formulation is particularly suitable
for mixed-integer quadratic programming (MIQP) extensions of fleet replacement problems.
When cost uncertainty is represented through a finite set of scenarios, the expected cost
is linear, while the variance term introduces a convex quadratic expression that can be
handled directly by modern solvers. This enables the systematic exploration of cost-risk
trade-offs using established multi-objective solution techniques, such as the
$\varepsilon$-constraint method, while maintaining computational tractability.

Overall, the mean–variance framework provides a coherent and well-founded approach for modeling risk in urban fleet replacement decisions. It allows the explicit representation of uncertainty, supports diversification across vehicle technologies, and offers decision-makers a transparent tool to balance economic performance and risk in the transition toward more sustainable urban freight systems.
