%!TEX root = ../template.tex
%%%%%%%%%%%%%%%%%%%%%%%%%%%%%%%%%%%%%%%%%%%%%%%%%%%%%%%%%%%%%%%%%%%%
%% chapter4.tex
%% NOVA thesis document file
%%
%% Chapter with lots of dummy text
%%%%%%%%%%%%%%%%%%%%%%%%%%%%%%%%%%%%%%%%%%%%%%%%%%%%%%%%%%%%%%%%%%%%

\typeout{NT FILE chapter4.tex}%

\chapter{Proposed Approach}
\label{cha:Proposed_Approach}

This chapter presents the proposed optimization framework for urban fleet replacement under uncertainty. The problem is formulated as a multi-objective optimization model that explicitly balances economic performance and cost risk over the planning period. Building on the mean-variance perspective introduced in Chapter~\ref{cha:literature_review}, vehicle technologies are treated as portfolio assets whose lifecycle costs exhibit different expected values and variances.

\section{Total Cost Formulation}
\label{sec:Total_Cost_Formulation}
The fleet replacement problem is formulated over a discrete planning period and a finite set of vehicle types, these vehicles consist of different technologies (ICEVs, EVs, HEVs) and different models for each technology. The main sets and indices used in the model are defined as follows:

\begin{itemize}
    \item $t \in \mathcal{T} = \{0, \dots, T\}$: index of time periods.
    \item $i \in \mathcal{I}_k = \{0, \dots, A_k\}$: index of vehicle age for type $k$.
    \item $k \in \mathcal{K} = \{1, \dots, K\}$: index of vehicle types.
\end{itemize}

The following decision variables are used in the optimization model:

\begin{itemize}
    \item $V_{i,t,k}$: number of vehicles of technology $k$ and age $i$ operated in period $t$.
    \item $P_{t,k}$: number of new vehicles of technology $k$ purchased at the beginning of period $t$.
    \item $S_{i,t,k}$: number of vehicles of technology $k$ and age $i$ salvaged at the end of period $t$.
\end{itemize}

All decision variables are non-negative integers, reflecting the discrete nature of fleet composition and replacement decisions.
The total cost formulation consists of the following components.

%% ------------------------- TOTAL ENERGY COST -------------------------
Energy Cost: The total energy cost incurred by operating the fleet is given by:
\begin{equation}
EC = \sum_{i=0}^{A_k-1}\sum_{t=0}^{T-1}\sum_{k=1}^{K}
f_{i,t,k}\,u_{i,k}\,V_{i,t,k}\,(1+dr)^{-t},
\label{eq:EC}
\end{equation}
where $f_{i,t,k}$ represents the unit energy cost per kilometer (\(\text{€}/\mathrm{km}\)) for a vehicle of age $i$ and technology $k$ in period $t$, $u_{i,k}$ is the annual utilization and $dr$ denotes the discount rate.

%% ----------------------- OPEX -------------------------
Maintenance Cost: Maintenance and operating costs are modeled as:
\begin{equation}
MC = \sum_{i=0}^{A_k-1}\sum_{t=0}^{T-1}\sum_{k=1}^{K}
m_{i,t,k}\,u_{i,k}\,V_{i,t,k}\,(1+dr)^{-t},
\label{eq:MC}
\end{equation}
where $m_{i,t,k}$ denotes the maintenance cost per kilometer of vehicle k of an age i during period t (€/km).

%% ----------------------- CAPEX -------------------------
Capital Investment Cost: Associated with acquiring new vehicles is expressed as:
\begin{equation}
CIC = \sum_{t=0}^{T-1}\sum_{k=1}^{K}
v_{k,t}\,P_{t,k}\,(1+dr)^{-t},
\label{eq:CIC}
\end{equation}
where $v_{k,t}$ is the purchase cost (€) per unit of a vehicle of technology $k$ in period $t$.

%% ----------------------- EMISSION COST -------------------------
Emission-related Cost:
\begin{equation}
ERC = \sum_{i=0}^{A_k-1}\sum_{t=0}^{T-1}\sum_{k=1}^{K}
e_{i,k}\,u_{i,k}\,V_{i,t,k}\,(1+dr)^{-t},
\label{eq:R}
\end{equation}
where $e_{i,k}$ represents CO$_2$ emission cost (€/km) of vehicle of age $i$, type $k$.

%% ----------------------- SALVAGE REVENUE -------------------------
Salvage Revenue: Obtained from retiring vehicles at the end of their service life is given by:
\begin{equation}
SR = \sum_{i=1}^{A_k}\sum_{t=0}^{T}\sum_{k=1}^{K}
s_{i,k}\,S_{i,t,k}\,(1+dr)^{-t},
\label{eq:SR}
\end{equation}
where $s_{i,k}$ denotes the salvage revenue (€) of a vehicle of age $i$ and technology $k$.

%% ----------------------- TOTAL COST -------------------------
Total Cost: By combining all cost components of the fleet over the planning period can be expressed as:

\begin{equation}
\begin{aligned}
TC =\;& \sum_{t=0}^{T-1}\sum_{k=1}^{K} v_{k,t}\,P_{t,k}\,(1+dr)^{-t} - \sum_{i=1}^{A_k}\sum_{t=0}^{T}\sum_{k=1}^{K} s_{i,k}\,S_{i,t,k}\,(1+dr)^{-t} \\
&+ \sum_{i=0}^{A_k-1}\sum_{t=0}^{T-1}\sum_{k=1}^{K}
\bigl(f_{i,t,k}+m_{i,t,k}+e_{i,k}\bigr)\,u_{i,k}\,V_{i,t,k}\,(1+dr)^{-t}.
\end{aligned}
\label{eq:TC1}
\end{equation}

$TC$ becomes a random variable when uncertainty is present in parameters such as energy prices, maintenance costs, or vehicle prices. This representation allows the Expected Total Cost and the variance of total cost to be computed across a finite set of scenarios, providing the basis for mean–variance optimization in fleet replacement decisions.

\section{Objective Functions}
\label{sec:objective_functions}

The problem considered is formulated as a multi-objective optimization problem that simultaneously accounts for economic efficiency and risk under uncertainty. Following the mean-variance framework originally developed in Modern Portfolio Theory and adapted to urban fleet management by Ahani, Arantes, and Melo~\cite{Ahani2016}, 

the objectives are defined as the minimization of the Expected Total Cost and the minimization of the Variance of the Total Cost over the planning period.
Let $TC_s$ denote the total discounted cost of the fleet under scenario $s$, computed according to the total cost formulation presented in Section~\ref{sec:Total_Cost_Formulation}. Consider a finite set of scenarios $s \in \mathcal{S}$, each associated with probability $\pi_s$.

The first objective is the minimization of the Expected Total Cost (ETC):

\begin{equation}
\min \; \mathbb{E}[TC] =
\sum_{s \in \mathcal{S}} \pi_s \, TC_s.
\label{eq:expected_total_cost}
\end{equation}

This objective captures the average economic performance of a fleet replacement strategy across all considered uncertainty scenarios. Minimizing the expected cost favors solutions that are economically efficient on average but does not account for the dispersion or volatility of cost outcomes.

To explicitly represent risk, the second objective minimizes the variance of the total cost across scenarios:

\begin{equation}
\min \; \mathrm{Var}(TC) =
\sum_{s \in \mathcal{S}} \pi_s
\left( TC_s - \mathbb{E}[TC] \right)^2 .
\label{eq:variance_total_cost}
\end{equation}

Although the scenario total cost $TC_s$ is linear in the decision variables, the variance
definition in ~\eqref{eq:variance_total_cost} contains squared deviation terms and therefore induces a
quadratic objective (or a quadratic constraint under the $\varepsilon$-constraint method).
Consequently, the resulting formulation is a mixed-integer quadratic program (MIQP).
When the variance is bounded by $\varepsilon$, the model becomes a mixed-integer
quadratically constrained program (MIQCP). This is consistent with \textcite{Ahani2016},
who solves the mean-variance fleet replacement formulation as a mixed-integer quadratic model.

The variance measures the dispersion of total cost outcomes around their expected value and serves as a proxy for cost risk. In the context of urban freight fleet planning, a higher variance indicates greater exposure to uncertain energy prices, technology costs, and other stochastic parameters, which may challenge budget stability and long-term investment planning.

Together, Equations~\eqref{eq:expected_total_cost} and~\eqref{eq:variance_total_cost} define a bi-objective optimization problem. Since the two objectives are generally conflicting, no single solution simultaneously minimizes both expected cost and risk. Instead, the problem admits a set of Pareto-optimal solutions, each representing a different trade-off between economic efficiency and risk exposure.

\section{Alternative Risk Measures (Model Variants)}
\label{sec:alt_risk_measures}

\subsection{Mean Absolute Deviation (MAD)}
\label{subsec:mad_formulation}

Variance leads to a quadratic risk term. A linear alternative is the Mean Absolute Deviation (MAD), which measures the expected absolute deviation of scenario costs from their mean. In a scenario-based setting, MAD can be embedded in the optimization model using auxiliary variables, yielding a mixed-integer linear program (MILP) variant \parencite{MANSINI2014518}. 
Let ${E}[TC]$ denote the expected total cost:
\[
{E}[TC] = \sum_{s\in\mathcal S}\pi_s\,TC_s.
\]
Define variables $u_s \ge 0$ such that:
\[
u_s \ge TC_s - {E}[TC],\quad u_s \ge -(TC_s - {E}[TC]),\quad \forall s\in\mathcal S.
\]
Then the MAD risk term is
\[
\sum_{s\in\mathcal S}\pi_s u_s.
\]

\subsection{Conditional Value-at-Risk (CVaR)}
\label{subsec:cvar_formulation}

To emphasize downside protection, Conditional Value-at-Risk (CVaR) captures the expected cost in the worst $(1-\alpha)$ tail of the distribution. Under a discrete scenario set, CVaR admits an equivalent formulation with linear constraints \parencite{RockafellarUryasev2000}. Introduce a variable $\eta$ (Value-at-Risk surrogate) and auxiliary variables $\xi_s \ge 0$:
\[
\xi_s \ge TC_s - \eta,\quad \xi_s \ge 0, \quad \forall s \in \mathcal{S}.
\]
Then
\[
\mathrm{CVaR}_{\alpha}(TC) = \eta + \frac{1}{1-\alpha}\sum_{s \in \mathcal{S}} \pi_s \xi_s.
\]
CVaR has also been applied to sustainable fleet replacement to quantify the risk of lifecycle costs under uncertain fuel and carbon-related parameters \parencite{ANSARIPOOR2014701}.


\section{Solution Approach: The $\varepsilon$-Constraint Method}
\label{sec:epsilon_constraint}

To solve the bi-objective fleet replacement problem, this thesis adopts the $\varepsilon$-constraint method, a widely used technique for multi-objective optimization. The core idea of the $\varepsilon$-constraint method is to optimize one objective while transforming the other objective into a constraint bounded by a predefined threshold.

The $\varepsilon$-constraint method is applicable not only to the baseline MIQP formulation but also to the MILP variants obtained under MAD or CVaR risk, enabling consistent construction of efficient frontiers across different risk definitions.

In this study, the Expected Total Cost is minimized while imposing an upper bound on the variance of total cost:

\begin{equation}
\min \; \mathbb{E}[TC]
\label{eq:epsilon_objective}
\end{equation}

subject to

\begin{equation}
\mathrm{Var}(TC) \leq \varepsilon,
\label{eq:epsilon_constraint}
\end{equation}

together with the fleet balance, acquisition, and operational constraints described in Section~\ref{sec:model_constraints}.

The parameter $\varepsilon$ represents the maximum level of cost variance that the decision-maker is willing to accept. By systematically varying $\varepsilon$ over a predefined range, multiple optimization problems are solved, each corresponding to a different risk tolerance level. The collection of resulting solutions approximates the efficient frontier, which characterizes the trade-offs between expected cost and risk.

From a decision-making perspective, low values of $\varepsilon$ correspond to risk-averse strategies that prioritize cost stability, potentially at the expense of higher expected costs. Conversely, larger values of $\varepsilon$ allow for more risk-taking behavior, favoring lower expected costs while accepting greater cost volatility. This structure enables fleet managers and policymakers to explicitly evaluate alternative replacement strategies and select solutions aligned with their risk preferences.

\section{Model Constraints}
\label{sec:model_constraints}

The optimization problem is subject to a set of constraints that govern fleet evolution over time, vehicle acquisition and retirement, and domain restrictions on decision variables.

\subsection{Fleet Balance and Aging Constraints}

Fleet dynamics are modeled through vehicle aging and replacement relationships. For each vehicle technology $k$, age $i \geq 1$, and period $t$, the number of vehicles of age $i$ in period $t+1$ is equal to the number of vehicles of age $i-1$ in period $t$ minus those that are salvaged:

\begin{equation}
V_{i,t+1,k} = V_{i-1,t,k} - P_{i-1,t,k},
\quad \forall i \geq 1, \; t, \; k.
\label{eq:aging_constraint}
\end{equation}

Newly acquired vehicles enter the fleet with age zero:

\begin{equation}
V_{0,t+1,k} = S_{t,k},
\quad \forall t, \; k.
\label{eq:new_vehicle_constraint}
\end{equation}

These constraints ensure consistency in fleet evolution and explicitly track the aging process of vehicles over the planning period.

\subsection{Acquisition and Retirement Constraints}

Vehicles can only be salvaged if they exist in the fleet:

\begin{equation}
P_{i,t,k} \leq V_{i,t,k},
\quad \forall i, \; t, \; k.
\label{eq:salvage_constraint}
\end{equation}

Additionally, vehicles that reach their maximum allowable age $A_k$ must be retired:

\begin{equation}
V_{A_k,t+1,k} = 0,
\quad \forall t, \; k.
\label{eq:max_age_constraint}
\end{equation}

These constraints prevent infeasible fleet configurations and ensure that vehicle lifetimes are respected.

\subsection{Domain Constraints}

All decision variables are restricted to non-negative integer values:

\begin{equation}
V_{i,t,k}, \; P_{i,t,k}, \; S_{t,k} \in \mathbb{S}_+,
\quad \forall i, \; t, \; k.
\label{eq:domain_constraints}
\end{equation}

This reflects the discrete nature of fleet composition and replacement decisions.

\section{Modeling Assumptions}
\label{sec:model_assumptions}

The proposed model assumes deterministic fleet demand and utilization levels over the planning period, while uncertainty is captured through stochastic cost parameters. Vehicle technologies are assumed to be perfectly substitutable in terms of service provision, and operational constraints such as routing or charging schedules, or depot capacity are not explicitly modeled, although these operational constraints could be added into the model to better reflect fleet operation in practice. These assumptions for now, allow the focus to remain on strategic fleet replacement decisions under uncertainty.


%% ---------------------------------------- FUTURE WORK CHAPTER 5 -------------------------------------------------- %%
\chapter{Future Work}
\label{cha:Future Work}

\section{Tasks}
\label{sec:Tasks}
Future work in this dissertation will include the following tasks:

\textbf{Task 1: Data Collection and Scenario Generation.} Collection of historical and projected data regarding vehicle acquisition costs (ICEVs, HEVs and EVs), energy prices (fuel and electricity), and maintenance costs. Based on this data, Python will be used to code the total cost formulation and the mean–variance objectives defined in Chapter 4, along with all operational and fleet balance constraints. The $\varepsilon$-constraint method will be applied to generate a set of Pareto-optimal solutions representing different trade-offs between Expected Total Cost and cost Variance. \par

Computational experiments will be conducted to assess model feasibility, convergence behavior, and solution quality under different parameter settings. These experiments will serve as the basis for evaluating the practical applicability of the proposed framework. \par

\textbf{Task 2: Analysis of Cost–Risk Trade-offs.} Once the efficient frontier is obtained, the next step involves analyzing the economic trade-offs between cost efficiency and risk exposure. The impact of increasing risk aversion on fleet composition will be examined by comparing solutions associated with different variance thresholds. This analysis will provide insights into how diversification across vehicle technologies contributes to reducing cost volatility while maintaining acceptable expected costs. \par

\textbf{Task 3: $\varepsilon$-constraint routine and efficient frontier construction.}
A dedicated routine will be implemented to solve a sequence of MIQCPs using the $\varepsilon$-constraint method. In each run, the model will minimize Expected Total Cost while enforcing a bound on the risk measure:
\begin{equation}
    \min \ \mathbb{E}[TC] \quad \text{s.t.} \quad \mathrm{Risk}(TC) \le \varepsilon.
\end{equation}
A grid (or adaptive set) of $\varepsilon$ values will be defined between a tight (risk-averse) bound and a loose (risk-tolerant) bound, generating a set of Pareto-efficient solutions. The outputs of this task will include (i) the efficient frontier, (ii) the corresponding fleet composition paths, and (iii) summary statistics for each solution.

\textbf{Task 4: Policy-Oriented Scenario Evaluation.} The proposed framework will also be used to explore policy-driven scenarios relevant to urban freight operations. These may include changes in energy taxation, technology-specific incentives, or access restrictions associated with low-emission zones. By adjusting cost parameters and constraints accordingly, the model can be used to assess how different policy environments influence fleet replacement decisions and the associated cost–risk trade-offs.

\textbf{Task 5: Model Extensions.} As a final step, potential extensions of the model will be explored, subject to time and data availability. These extensions may include a more detailed representation of fleet heterogeneity, such as multiple vehicle sizes or differentiated usage profiles, as well as alternative planning horizons. While not essential for the core objectives of this dissertation, such extensions would further enhance the realism and applicability of the proposed framework.

\textbf{Task 6: Writing the Dissertation.} The implemented methodology and computational results will be integrated into the dissertation. This includes documenting the data sources and scenario design, presenting the efficient frontier and solution interpretation, and formalizing limitations and extensions. Additionally, compilation of all chapters, including the literature review, methodology, discussion of results, and conclusions. This task also includes the formatting of the document according to the university's standards and the preparation for the final defense.

\section{Schedule}
Figure~\ref{fig:schedule} shows the expected schedule for the tasks proposed in Section~\ref{sec:Tasks}
\begin{figure}[htbp]
    \centering
    \includegraphics[width=1\linewidth]{schedule}
    \caption{Projected tasks schedule}
    \label{fig:schedule}
\end{figure}