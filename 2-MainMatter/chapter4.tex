%!TEX root = ../template.tex
%%%%%%%%%%%%%%%%%%%%%%%%%%%%%%%%%%%%%%%%%%%%%%%%%%%%%%%%%%%%%%%%%%%%
%% chapter4.tex
%% NOVA thesis document file
%%
%% Chapter with lots of dummy text
%%%%%%%%%%%%%%%%%%%%%%%%%%%%%%%%%%%%%%%%%%%%%%%%%%%%%%%%%%%%%%%%%%%%

\typeout{NT FILE chapter4.tex}%

\chapter{Proposed Approach}
\label{cha:Proposed_Approach}

This chapter presents the proposed optimization framework for urban fleet replacement under uncertainty. The problem is formulated as a multi-objective optimization model that explicitly balances economic performance and cost risk over a multi-period planning horizon. Building on the mean--variance perspective introduced in Chapter~\ref{cha:literature_review}, vehicle technologies are treated as portfolio assets whose lifecycle costs exhibit different expected values and variances.

\section{Objective Functions}
\label{sec:objective_functions}

The urban fleet replacement problem considered in this thesis is formulated as a multi-objective optimization problem that simultaneously accounts for economic efficiency and risk under uncertainty. Following the mean--variance framework originally developed in Modern Portfolio Theory and adapted to urban fleet management by Ahani, Arantes, and Melo~\cite{Ahani2016}, the objectives are defined as the minimization of the Expected Total Cost and the minimization of the Variance of the Total Cost over the planning horizon.

Let $TC(s)$ denote the total discounted cost of the fleet under scenario $s$, computed according to the total cost formulation presented in Section~\ref{sec:total_cost_formulation}. Consider a finite set of scenarios $s \in \mathcal{S}$, each associated with probability $\pi_s$.

The first objective is the minimization of the Expected Total Cost (ETC):

\begin{equation}
\min \; \mathbb{E}[TC] =
\sum_{s \in \mathcal{S}} \pi_s \, TC(s).
\label{eq:expected_total_cost}
\end{equation}

This objective captures the average economic performance of a fleet replacement strategy across all considered uncertainty scenarios. Minimizing the expected cost favors solutions that are economically efficient on average but does not account for the dispersion or volatility of cost outcomes.

To explicitly represent risk, the second objective minimizes the variance of the total cost across scenarios:

\begin{equation}
\min \; \mathrm{Var}(TC) =
\sum_{s \in \mathcal{S}} \pi_s
\left( TC(s) - \mathbb{E}[TC] \right)^2 .
\label{eq:variance_total_cost}
\end{equation}

The variance measures the dispersion of total cost outcomes around their expected value and serves as a proxy for cost risk. In the context of urban freight fleet planning, a higher variance indicates greater exposure to uncertain energy prices, technology costs, and other stochastic parameters, which may challenge budget stability and long-term investment planning.

Together, Equations~\eqref{eq:expected_total_cost} and~\eqref{eq:variance_total_cost} define a bi-objective optimization problem. Since the two objectives are generally conflicting, no single solution simultaneously minimizes both expected cost and risk. Instead, the problem admits a set of Pareto-optimal solutions, each representing a different trade-off between economic efficiency and risk exposure.

\section{Solution Approach: The $\varepsilon$-Constraint Method}
\label{sec:epsilon_constraint}

To solve the bi-objective fleet replacement problem, this thesis adopts the $\varepsilon$-constraint method, a widely used technique for multi-objective optimization. The core idea of the $\varepsilon$-constraint method is to optimize one objective while transforming the other objective into a constraint bounded by a predefined threshold.

In this study, the Expected Total Cost is minimized while imposing an upper bound on the variance of total cost:

\begin{equation}
\min \; \mathbb{E}[TC]
\label{eq:epsilon_objective}
\end{equation}

subject to

\begin{equation}
\mathrm{Var}(TC) \leq \varepsilon,
\label{eq:epsilon_constraint}
\end{equation}

together with the fleet balance, acquisition, and operational constraints described in Section~\ref{sec:model_constraints}.

The parameter $\varepsilon$ represents the maximum level of cost variance that the decision-maker is willing to accept. By systematically varying $\varepsilon$ over a predefined range, multiple optimization problems are solved, each corresponding to a different risk tolerance level. The collection of resulting solutions approximates the efficient frontier, which characterizes the trade-offs between expected cost and risk.

From a decision-making perspective, low values of $\varepsilon$ correspond to risk-averse strategies that prioritize cost stability, potentially at the expense of higher expected costs. Conversely, larger values of $\varepsilon$ allow for more risk-taking behavior, favoring lower expected costs while accepting greater cost volatility. This structure enables fleet managers and policymakers to explicitly evaluate alternative replacement strategies and select solutions aligned with their risk preferences.

\section{Model Constraints}
\label{sec:model_constraints}

The optimization problem is subject to a set of constraints that govern fleet evolution over time, vehicle acquisition and retirement, and domain restrictions on decision variables.

\subsection{Fleet Balance and Aging Constraints}

Fleet dynamics are modeled through vehicle aging and replacement relationships. For each vehicle technology $k$, age $i \geq 1$, and period $t$, the number of vehicles of age $i$ in period $t+1$ is equal to the number of vehicles of age $i-1$ in period $t$ minus those that are salvaged:

\begin{equation}
X_{i,t+1,k} = X_{i-1,t,k} - Y_{i-1,t,k},
\quad \forall i \geq 1, \; t, \; k.
\label{eq:aging_constraint}
\end{equation}

Newly acquired vehicles enter the fleet with age zero:

\begin{equation}
X_{0,t+1,k} = Z_{t,k},
\quad \forall t, \; k.
\label{eq:new_vehicle_constraint}
\end{equation}

These constraints ensure consistency in fleet evolution and explicitly track the aging process of vehicles over the planning horizon.

\subsection{Acquisition and Retirement Constraints}

Vehicles can only be salvaged if they exist in the fleet:

\begin{equation}
Y_{i,t,k} \leq X_{i,t,k},
\quad \forall i, \; t, \; k.
\label{eq:salvage_constraint}
\end{equation}

Additionally, vehicles that reach their maximum allowable age $A_k$ must be retired:

\begin{equation}
X_{A_k,t+1,k} = 0,
\quad \forall t, \; k.
\label{eq:max_age_constraint}
\end{equation}

These constraints prevent infeasible fleet configurations and ensure that vehicle lifetimes are respected.

\subsection{Domain Constraints}

All decision variables are restricted to non-negative integer values:

\begin{equation}
X_{i,t,k}, \; Y_{i,t,k}, \; Z_{t,k} \in \mathbb{Z}_+,
\quad \forall i, \; t, \; k.
\label{eq:domain_constraints}
\end{equation}

This reflects the discrete nature of fleet composition and replacement decisions.

\section{Modeling Assumptions}
\label{sec:model_assumptions}

The proposed model assumes deterministic fleet demand and utilization levels over the planning horizon, while uncertainty is captured through stochastic cost parameters. Vehicle technologies are assumed to be perfectly substitutable in terms of service provision, and operational constraints such as routing or charging schedules, or depot capacity are not explicitly modeled, although these operational constraints could be added into the model to better reflect fleet operation in practice. These assumptions for now, allow the focus to remain on strategic fleet replacement decisions under uncertainty.


%% ---------------------------------------- FUTURE WORK CHAPTER 5 -------------------------------------------------- %%
\chapter{Future Work}
\label{cha:Future Work}

\section{Tasks}
\label{sec:Tasks}
Future work in this dissertation will include the following tasks:

\textbf{Task 1: Data Collection and Scenario Generation.} Collection of historical and projected data regarding vehicle acquisition costs (ICEVs, HEVs and EVs), energy prices (fuel and electricity), and maintenance costs. Based on this data, I will use Python to code the total cost formulation and the mean–variance objectives defined in Chapter 4, along with all operational and fleet balance constraints. The $\varepsilon$-constraint method will be applied to generate a set of Pareto-optimal solutions representing different trade-offs between Expected Total Cost and cost Variance. \par

Computational experiments will be conducted to assess model feasibility, convergence behavior, and solution quality under different parameter settings. These experiments will serve as the basis for evaluating the practical applicability of the proposed framework. \par

\textbf{Task 2: Analysis of Cost–Risk Trade-offs.} Once the efficient frontier is obtained, the next step involves analyzing the economic trade-offs between cost efficiency and risk exposure. The impact of increasing risk aversion on fleet composition will be examined by comparing solutions associated with different variance thresholds. This analysis will provide insights into how diversification across vehicle technologies contributes to reducing cost volatility while maintaining acceptable expected costs. \par

\textbf{Task 3: Sensitivity and Elasticity Analysis.} Conducting the robustness analysis defined in Research Question 4. I will systematically vary critical parameters (electricity prices, battery degradation costs, and carbon taxes) to calculate the elasticity of the fleet composition. The goal is to identify the economic "tipping points" where the optimal decision shifts significantly from ICEVs to EVs.\par

\textbf{Task 4: Policy-Oriented Scenario Evaluation.} The proposed framework will also be used to explore policy-driven scenarios relevant to urban freight operations. These may include changes in energy taxation, technology-specific incentives, or access restrictions associated with low-emission zones. By adjusting cost parameters and constraints accordingly, the model can be used to assess how different policy environments influence fleet replacement decisions and the associated cost–risk trade-offs.

\textbf{Task 5: Model Extensions.} As a final step, potential extensions of the model will be explored, subject to time and data availability. These extensions may include a more detailed representation of fleet heterogeneity, such as multiple vehicle sizes or differentiated usage profiles, as well as alternative planning horizons. While not essential for the core objectives of this dissertation, such extensions would further enhance the realism and applicability of the proposed framework.

\textbf{Task 6: Writing the Dissertation.} Compilation of all chapters, including the literature review, methodology, discussion of results, and conclusions. This task also includes the formatting of the document according to the university's standards and the preparation for the final defense.

\section{Schedule}
Figure~\ref{fig:schedule} shows the expected schedule for the tasks proposed in Section~\ref{sec:Tasks}
\begin{figure}[htbp]
    \centering
    \includegraphics[width=1\linewidth]{schedule}
    \caption{Projected tasks schedule}
    \label{fig:schedule}
\end{figure}