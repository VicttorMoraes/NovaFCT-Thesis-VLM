%!TEX root = ../template.tex
%%%%%%%%%%%%%%%%%%%%%%%%%%%%%%%%%%%%%%%%%%%%%%%%%%%%%%%%%%%%%%%%%%%%
%% abstract-en.tex
%% NOVA thesis document file
%%
%% Abstract in English
%%%%%%%%%%%%%%%%%%%%%%%%%%%%%%%%%%%%%%%%%%%%%%%%%%%%%%%%%%%%%%%%%%%%

\typeout{NT FILE abstract-en.tex}%

Urban freight operators face increasing pressure to decarbonize while managing long-lived assets under volatile and uncertain cost drivers such as fuel and electricity prices, technology cost evolution, and residual values. This dissertation proposes a decision-support framework for urban fleet replacement planning that explicitly balances economic performance and cost risk under uncertainty. Building on a mean–variance portfolio perspective, vehicle technologies are treated as ``assets'' whose lifecycle costs exhibit distinct expected values and variances.

The planning problem is formulated as a bi-objective mixed-integer optimization model that minimizes (i) the \emph{Expected Total Cost} over a multi-period horizon and (ii) the \emph{Variance of Total Cost} across a discrete set of uncertainty scenarios. Fleet dynamics are represented through vehicle aging, acquisition, and retirement constraints, enabling heterogeneous technologies (e.g., diesel, electric, hybrid) and potentially multiple vehicle classes. To characterize the efficient set of cost-risk trade-offs, the $\varepsilon$-constraint method is adopted: the expected cost is minimized while imposing an upper bound $\varepsilon$ on the cost variance, and $\varepsilon$ is varied systematically to generate Pareto-optimal replacement plans.

A case study is planned for an urban operator in Lisbon (public or private). If proprietary operational data cannot be obtained, the model will be parameterized using data from the literature and publicly available sources. The resulting efficient frontier is expected to support managerial and policy analysis by revealing how risk tolerance, uncertainty in energy prices, and technology trajectories influence optimal fleet composition and replacement timing.

\keywords{
Urban fleet replacement \and
mean–variance \and
Risk under uncertainty \and
Multi-objective optimization \and
$\varepsilon$-constraint method
}


% Regardless of the language in which the dissertation is written, usually there are at least two abstracts: one abstract in the same language as the main text, and another abstract in some other language.

% The abstracts' order varies with the school.  If your school has specific regulations concerning the abstracts' order, the \gls{novathesis} (\LaTeX) template will respect them.  Otherwise, the default rule in the \gls{novathesis} template is to have in first place the abstract in \emph{the same language as main text}, and then the abstract in \emph{the other language}. For example, if the dissertation is written in Portuguese, the abstracts' order will be first Portuguese and then English, followed by the main text in Portuguese. If the dissertation is written in English, the abstracts' order will be first English and then Portuguese, followed by the main text in English.
% %
% However, this order can be customized by adding one of the following to the file \verb+5_packages.tex+.

% \begin{verbatim}
%     \ntsetup{abstractorder={<LANG_1>,...,<LANG_N>}}
%     \ntsetup{abstractorder={<MAIN_LANG>={<LANG_1>,...,<LANG_N>}}}
% \end{verbatim}

% For example, for a main document written in German with abstracts written in German, English and Italian (by this order) use:
% \begin{verbatim}
%     \ntsetup{abstractorder={de={de,en,it}}}
% \end{verbatim}

% Concerning its contents, the abstracts should not exceed one page and may answer the following questions (it is essential to adapt to the usual practices of your scientific area):

% \begin{enumerate}
%   \item What is the problem?
%   \item Why is this problem interesting/challenging?
%   \item What is the proposed approach/solution/contribution?
%   \item What results (implications/consequences) from the solution?
% \end{enumerate}

% % Abstract keywords
% \keywords{
%   TESTE One keyword \and
%   Another keyword \and
%   Yet another keyword \and
%   One keyword more \and
%   The last keyword
% }
