%!TEX root = ../template.tex
%%%%%%%%%%%%%%%%%%%%%%%%%%%%%%%%%%%%%%%%%%%%%%%%%%%%%%%%%%%%%%%%%%%%
%% abstract-pt.tex
%% NOVA thesis document file
%%
%% Abstract in Portuguese
%%%%%%%%%%%%%%%%%%%%%%%%%%%%%%%%%%%%%%%%%%%%%%%%%%%%%%%%%%%%%%%%%%%%

\typeout{NT FILE abstract-pt.tex}%

Operadores de transporte urbano de mercadorias enfrentam pressão crescente para descarbonizar as suas operações, ao mesmo tempo que gerem ativos de longa duração sujeitos a incerteza em fatores de custo como preços de combustíveis e eletricidade, evolução dos custos tecnológicos e valores residuais. Esta dissertação propõe um modelo de apoio à decisão para o planeamento de substituição de frota urbana que equilibra explicitamente desempenho económico e risco de custo sob incerteza. Com base numa perspetiva mean–variance inspirada na teoria de carteiras, as tecnologias de veículos são tratadas como "ativos" cujos custos de ciclo de vida apresentam diferentes valores esperados e variâncias.

O problema é formulado como um modelo de otimização bi-objetivo com variáveis inteiras, minimizando (i) o \emph{Custo Total Esperado} num horizonte multi-período e (ii) a \emph{Variância do Custo Total} calculada sobre um conjunto discreto de cenários de incerteza. A dinâmica da frota é representada por restrições de envelhecimento, aquisição e abate, permitindo considerar múltiplas tecnologias (diesel, elétrica e híbrida) e, potencialmente, diferentes classes de veículos. Para obter o conjunto eficiente de \emph{trade-offs} custo-risco, adota-se o método $\varepsilon$-constraint: minimiza-se o custo esperado impondo um limite superior $\varepsilon$ para a variância, variando-se $\varepsilon$ de forma sistemática para gerar planos de substituição Pareto-ótimos.

Está previsto um estudo de caso com um operador urbano em Lisboa (público ou privado). Caso não seja possível obter dados proprietários, o modelo será parametrizado com dados da literatura e fontes públicas. A fronteira eficiente resultante deverá apoiar análises de gestão e de política pública, mostrando como a tolerância ao risco, a incerteza nos preços de energia e as trajetórias tecnológicas influenciam a composição ótima da frota e o calendário de substituição.

\keywords{
Substituição de frota urbana \and
Transporte urbano de mercadorias \and
Variância-Média \and
Risco sob incerteza \and
Otimização multiobjetivo \and
Método $\varepsilon$-constraint \and
}
% to add an extra black line
